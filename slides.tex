%\PassOptionsToPackage{cmyk}{xcolor}
%\pdfpageattr {/Group << /S /Transparency /I false /CS /DeviceRGB>>}
\documentclass[english,aspectratio=1610,smaller]{beamer}

%\usepackage[pdftex]{graphicx}

%\usepackage[absolute,overlay]{textpos}
%\setlength{\TPHorizModule}{1cm}
%\setlength{\TPVertModule}{1cm}

\usepackage{amssymb,amsmath}

\usepackage{listings}
\usepackage[normalem]{ulem}

\definecolor{grey}{rgb}{.5,.5,.5}
\definecolor{red}{rgb}{.9,0,0}
\definecolor{blue}{rgb}{0,0,.75}
\definecolor{green}{rgb}{0.0,.5,0.0}
%\definecolor{green}{rgb}{0,.6,0}
\definecolor{magenta}{cmyk}{0,1,0,0}
\definecolor{myyellow}{rgb}{.98,.84,.37}
\newcommand{\black}{\color{black}}
\newcommand{\white}{\color{white}}
\newcommand{\grey}{\color{grey}}
\newcommand{\red}{\color{red}}
\newcommand{\blue}{\color{blue}}
\newcommand{\green}{\color{green}}
\newcommand{\magenta}{\color{magenta}}
\newcommand{\orange}{\color{orange}}

\DeclareMathOperator{\sgn}{sign}
\DeclareMathOperator{\re}{Re}
\DeclareMathOperator{\im}{Im}
\DeclareMathOperator{\myspan}{span}
\DeclareMathOperator{\Gl}{Gl}
\DeclareMathOperator{\tr}{Tr}
\DeclareMathOperator{\diag}{diag}
\DeclareMathOperator{\kout}{k}
\DeclareMathOperator{\kin}{l}


\lstset{language=C++,
                basicstyle=\ttfamily,
                keywordstyle=\color{blue}\ttfamily,
                stringstyle=\color{red}\ttfamily,
                commentstyle=\color{green}\ttfamily,
                morecomment=[l][\color{magenta}]{\#}
}
\lstset{language=C++,
                basicstyle=\ttfamily
}

\definecolor{listinggray}{gray}{0.9}
\definecolor{lbcolor}{rgb}{0.9,0.9,1.0}

\lstset{
backgroundcolor=\color{lbcolor},
    tabsize=4,        
    language=Python,
    basicstyle=\scriptsize,
    columns=fixed,
    showstringspaces=false,
    extendedchars=false,
    breaklines=false,
    breakindent=2cm,
    prebreak = \raisebox{0ex}[0ex][0ex]{\ensuremath{\hookleftarrow}},
    frame=single,
    numbers=left,
    showtabs=false,
    showspaces=false,
    showstringspaces=false,
    linewidth=\textwidth,
    xleftmargin=0.04\textwidth,
    identifierstyle=\ttfamily,
    keywordstyle=\color[rgb]{0,0.5,0},
    commentstyle=\color[rgb]{0.6,0.1,0.0},
    stringstyle=\color[rgb]{0.627,0.126,0.941},
    numberstyle=\color[rgb]{0.205, 0.142, 0.73},
    otherkeywords={__m512, __mmask16, complex, override, size_t},
}

%\usetheme{default}
%\usetheme{CambridgeUS}
\usepackage{../ess-beamer-simon/beamerthemeESS}
\setbeamertemplate{navigation symbols}{}
%\setbeamertemplate{footline}[frame number]
\addtobeamertemplate{navigation symbols}{}{%
    \usebeamerfont{footline}%
    \usebeamercolor[fg]{footline}%
    \hspace{1em}%
    \insertframenumber/\inserttotalframenumber
}

\newcommand{\backupbegin}{
   \newcounter{finalframe}
   \setcounter{finalframe}{\value{framenumber}}
}
\newcommand{\backupend}{
   \setcounter{framenumber}{\value{finalframe}}
}

\graphicspath{{../../code/dataset/doc/},{./figs/}}

\usepackage{xspace}
\newcommand{\scipp}{\texttt{scipp}\xspace}
\newcommand{\scippneutron}{\texttt{scippneutron}\xspace}
\newcommand{\scippnexus}{\texttt{scippnexus}\xspace}
\newcommand{\pip}{\texttt{pip}\xspace}
\newcommand{\conda}{\texttt{conda}\xspace}

\begin{document}

\title{Dependency management: Lessons learned}
\author[Simon Heybrock]{Simon Heybrock \\{\footnotesize\url{simon.heybrock@ess.eu}}\\\vspace{2mm}
}
%\institute[ESS]{European Spallation Source}

\date{\scriptsize DMSC --- 2022-05-11}


\begin{frame}[plain]
    \titlepage
\end{frame}


\section{Dependency management: We must be doing something wrong?}

\begin{frame}{Context}
  \begin{itemize}
    \item Developing Python and/or C++ libraries and applications
    \item Using GitHub (or similar) with Github Actions (or similar) to:
      \begin{itemize}
        \item Build and run tests for pull requests (PRs).
        \item Build documentation.
        \item Build conda packages.
        \item Build PyPI wheels.
      \end{itemize}
  \end{itemize}
\end{frame}

\begin{frame}{Problem}
  \begin{block}{Example: scipp, scippnexus, scippneutron, ess}
    \begin{itemize}
      \item Experienced issues in about 10 cases in 2021, such as:
        \begin{itemize}
          \item Pull request builds suddenly fail for no apparent reason.
          \item Documentation builds stop working.
          \item Package builds don't work any more.
        \end{itemize}
      \item \alert{\dots despite previously passing CI}, issue is \emph{not} a bad merge.
    \end{itemize}
  \end{block}
  \begin{block}{Previous approach}
    \begin{itemize}
      \item Waste time discussing the ``weird'' issue at the daily standup.
      \item Dig through build logs: which package upgraded?
      \item Does older version work? Check upstream for issue, or open one.
      \item Pin version, restart all builds.
      \item Open issue as a reminder to remove the pin once upstream made a fix.
      \item Add pin across all projects.
      \item Remove pins once possible, half the time it is unclear how to reproduce.
    \end{itemize}
  \end{block}
\end{frame}

\begin{frame}[fragile]{Concrete examples from 2022}
  \begin{block}{\emph{Five} such issues is less than four months!}
    \begin{description}
      \item[February 18] Conan suddenly produces an error, caused by update of \texttt{markupsafe} \url{https://github.com/pallets/markupsafe/issues/282}.
        Lost ability to make a (patch) release until we found this and pinned the version.
      \item[March 24] Documentation builds stop working due to \url{https://github.com/jupyter/nbconvert/issues/1736} causes by \url{https://github.com/pallets/jinja/issues/1626}.
        Lost ability to pass CI (blocking all merges) as well as blocking release builds (which build and update docs).
      \item[April 1] We notice that navigation buttons in documentation pages have disappeared. The \texttt{sphinx-book-theme} 0.3.0 release causes this.
      \item[April 7] \texttt{scikit-build} releases 0.14.0, breaking our PyPI builds.
        Lost ability to make a (patch) release until we found this and pinned the version.
      \item[April 19] We notice that plots no longer show up in documentation pages, traced to an \texttt{ipympl} update \url{https://github.com/matplotlib/ipympl/issues/462}.
    \end{description}
  \end{block}
  By continuously upgrading, we experience \emph{almost every upstream issue}, even if it is fixed without hours or days.
\end{frame}

\section{Our way forward}

\begin{frame}{Recommended reading}
  \begin{block}{Approach}
    \begin{itemize}
      \item Step 1: Read \url{https://hynek.me/articles/semver-will-not-save-you/}. Recommends to freeze everything, including transitive dependencies.
      \item Step 2: \alt<2>
        {\sout{Conclude that we cannot do that because if we do that for a \emph{library} almost no one will be able to use it.}}
        {Conclude that we cannot do that because if we do that for a \emph{library} almost no one will be able to use it.}
      \item<2-> Step 3: Realize that many or most of our dependency issues are not issues that ever reach a user.
        Actually we can freeze for quite a few cases:
        \begin{itemize}
          \item Building documentation.
          \item Requirements need for building packages.
          \item Pull-request builds.
        \end{itemize}
      \item<2-> Step 4: Must read for more details and better understanding:
        \url{https://iscinumpy.dev/post/bound-version-constraints/}
    \end{itemize}
  \end{block}
\end{frame}


\begin{frame}{Implementation}
  \begin{block}{Requirement: All builds should be \emph{reproducible}}
    Reproducible means:
    \begin{itemize}
      \item Builds for a commit that passed previously should continue passing if we re-run it later.
      \item Builds should fail \emph{only} when there is a problem in the \emph{current} source repository, \emph{not} if a dependency or transitive dependency makes a release.
    \end{itemize}
    This requirement applies to ``everything'':
    \begin{itemize}
      \item Going back to much older version, e.g., to make a patch release.
      \item Documentation builds (Sphinx).
      \item Release builds (making packages for PyPI and conda).
      \item PR builds.
      \item Builds on \texttt{main} after merging a PR.
      \item \dots
    \end{itemize}
  \end{block}
\end{frame}

\begin{frame}{Workflow}
  \begin{block}{Setup (example for Python-based project with \texttt{pip} workflow)}
    \begin{enumerate}
      \item Write \texttt{requirements.in} file(s).
        Do this for \emph{everything} this is needed for your builds.
      \item Use \texttt{pip-compile} from \texttt{pip-tools}. See also \texttt{pip-compile-multi}.
        Resulting \texttt{requirements.txt} is \emph{checked into version control}.
      \item Treat \emph{warnings as errors} in CI, in particular \texttt{DeprecationWarning}
        \begin{itemize}
          \item Avoids issues from \emph{intentional} breaking changes, since we can take action early.
        \end{itemize}
    \end{enumerate}
  \end{block}
  \begin{block}{Continuously and from time-to-time}
    \begin{itemize}
      \item Re-run \texttt{pip-compile}, run all the builds (including release builds!) and check that things work, merge.
        \begin{itemize}
          \item \texttt{dependabot} may be an alternative, but it updates individual requirements, which seems too noisy for us, with too little control.
        \end{itemize}
      \item Make patch-releases when a dependency update causes issues for users
        \begin{itemize}
          \item Much rarer than issues we experience in development.
          \item Treating \texttt{DeprecationWarning} as errors should catch many issues months/years ahead.
          \item Users can freeze dependency version \emph{when setting up their environment} as a immediate workaround.
        \end{itemize}
    \end{itemize}
  \end{block}
\end{frame}

\begin{frame}{Status}
  \begin{itemize}
    \item We (scipp and subprojects) do \emph{not} have long term experience with this yet:
      \begin{itemize}
        \item So far implemented in \url{https://github.com/scipp/scippnexus} and \url{https://github.com/scipp/scipp} (excluding C++ build dependencies).
        \item Have not gone through update cycles (re-run of \texttt{pip-compile}) and do not know how many headaches this will cause.
      \end{itemize}
    \item Our \texttt{conan} dependencies have been pinned since the beginning.
    \item \texttt{conda} dependencies are not pinned yet. What is best use of tools for keeing this up to date, including pins of transitive dependencies?
  \end{itemize}
\end{frame}

\begin{frame}{Case study: Should we freeze dependencies in PR builds?}
  %To be alerted early of upstream changes that break our package, we may be tempted to always use the latest released version.
  %For example, the PR jobs that run the unit tests for \scippnexus could use the latest \scipp.
  \begin{block}{Temptation: Use latest version of library dependencies to be alerted early of upstream changes}
    Would this be a good idea?
    \pause
  \begin{itemize}[<+->]
    \item A new contributor, Alice, opens a PR with a small bugfix.
    \item The builds do not pass, since a completely unrelated test is failing (because there was a unrelated dependency update).
      Alice is confused.
    \item Bob, the project maintainer, figures out that what the issue is, explains it in the PR, fixes the issue, merges into \texttt{main}, merges \texttt{main} into the PR, explains that all is good now, \dots
    \item Alice is frustrated by the delay and noise and moves on.
    \item Bob is frustrated because he has to do things like this all the time.
  \end{itemize}
  \end{block}
  \pause
  \begin{block}{Better solution: Freeze all dependencies for PR builds to \emph{decouple} the tasks}
    \begin{itemize}
      \item A new contributor, Alice, opens a PR with a small bugfix. Builds pass and it gets merged. Alice is happy and may continue contributing.
      \item \emph{Independently}, Bob notices that a new upstream release requires a fix.
        He re-runs \texttt{pip-compile} and fixes the issue in a PR.
        After a merge, all future PR builds will now use the new upstream version.
    \end{itemize}
  \end{block}
\end{frame}

\end{document}
